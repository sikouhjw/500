\begin{jianjie}
  本书是为了有效地提高学生求解高等数学证明题的效率,培养训练数学思想方法与掌握数学算理,引导学生探索证明题的基本求解思路。怎样寻找有效途径可以达到证明目的?如果题目的已知条件不变化,而证明的结论发生变化,证明的思路将发生什么变化?如果已知条件变化,而证明的结论不变,证明的思路将发生什么变化?外观形式相仿的题目,证明的思路是否相同?外观形式不同的证明题,它们的证明思路是否也不同?希望能通过这种训练,有效地提高证明题的求解能力。

  本书选题范围较广。依据高等数学教学基本要求,参考研究生人学数学考试大纲,由多本高等数学习题集、考研试题、数学竞赛题中选择约 500 道证明题进行归类、分析。
  
  本书适用于理工类、经济类、管理类本科生学习,也适用于备考研究生的学生选作学习证明题的参考书。
\end{jianjie}

\chapter{前言}
  学习高等数学,要求学生掌握本学科的基本概念、基本性质和基本方法。进一步还要求学生掌握本学科的知识体系、知识框架,期望学生通过学习高等数学,提高抽象思维能力、逻辑推理能力、空间想像能力、运算能力和运用所学知识分析问题和解决问题的能力。学习数学证明题是学习数学过程中的重要环节之一。数学证明问题通常是检查学生对基本知识掌握程度的重要手段,也是培养学生各种能力的有效方法之一。

  有效地提高解答数学证明题的效率是学生共同的目标,也是数学教师普遍关心的问题。多年来经常看到有些数学习题集前后相隔很远的地方出现的题目,虽然外观形式差异较大,但实质是同一类题目,证明思路完全相同。学生常常是给出了前面题目的证明,但是不知道后面的题目如何下手?有些考试试题或数学竞赛题中出现的题目,是习题集中某个题目的特殊情形或推广形式,但是考生得分率很低。这从某种程度上说明学生有个共性问题:需要学习数学证明题的求解基本思想、需要学习掌握数学算理。

  本书期望能解决上述问题,引导学生发掘更深层次的问题,本书的主要特色为
  \begin{enumerate}
    \item 对所选高等数学证明题进行对比、分类、归纳
    
    将证明思路相同的题目、证明结论相同的题目、已知条件相同的题目等集中对比,归纳,以引起读者注意证明的基本思想有何变化?希望引导学生从这些数学证明问题的常见方法中,学习发现数学的基本算理,培养训练数学思想方法,本书立意引导学生思考所给问题的证明思路是什么?怎样寻找有效途径得到所要证明的结论?如果题目的已知条件不变化,而证明的结论发生变化,证明的思路将发生什么变化?如果已知条件发生变化,而证明的结论不变,证明的思路将发生什么变化?外观形式相仿的题目,是否证明思路相同?外观形式不同的题目,是否证明的思路也不同?本书希望读者通过这种训练,有效地提高证明题的求解能力,打牢数学基础。
    \item 选题范围较广
    
    依据高等数学教学基本要求,参考研究生入学数学考试大纲,本书选题参考同济大学《高等数学习题集》,吉米多维奇《数学分析习题集》,全国研究生入学考试数学试题,北京市及部分省市大学生数学竞赛试题,美国普特南数学竞赛题及俄罗斯大学生数学竞赛题。从中选择约 500 题,加以对比、分类、归纳梳理。

    例如,书中~\ref{sec:1.2.2} 第~\ref{sec:1.2.2-6} 题所给四个题设条件不变,而要证的结论发生变化。分析所给四个题目的特点,可以发现这几个题证明思想相同,当属同一类题目。

    再如书中~\ref{sec:1.1.1} 第~\ref{sec:1.1.1-13} 题所给三个题目的外观相似,证明思想相同,但是证明的方法有较大的差异。

    书中~\ref{sec:1.2.3} 第~\ref{sec:1.2.3-20} 题所给五个题目的题设条件不同,但证明的结论相同。由于题目的已知条件差异较大,各题的证明思想也相差较大。

    \ref{sec:1.3.2} 第~\ref{sec:1.3.2-28} 题中九个题的条件、结论都发生了某些变化,但是它们的证明思想是相同的。

    本书将一些形似或方法相似的题目归纳到一起,意图引导读者对照学习,找出不同题目之间的共性与差异,学习分析证明方法,从而有效地提高求解证明题的能力。

    本书为高等教育出版社出版的学习辅导系列之一。该系列辅导书分别从概念与性质、基本运算、证明题三个侧面编写。包括
    \begin{xiaoti}
      \item 《高等数学证明题 500 例解析》,由北京航空航天大学徐兵教授编写。
      \item 《线性代数、概率论与数理统计证明题 500 例解析》,由南开大学肖马成教授、周概容教授编写。
      \item 《考研数学、焦点概念与性质》,由徐兵、肖马成、周概容编写。
      \item 《考研数学历年真题解析与应试对策》(理工类),由徐兵、肖马成、周概容编写。
      \item 《考研数学历年真题解析与应试对策》(经济类),由徐兵、肖马成、周概容编写。
      
      从某种意义上说,《考研数学焦点概念与性质》是针对概念与性质的专项辅导书;《考研数学历年真题解析与应试对策》是对提高运算能力有特殊作用的辅导书;《高等数学证明题 500 例解析》与《线性代数、概率论与数理统计证明题 500 例解析》是对提高求解证明题能力有明显功效的辅导书。可以说对于大学生学习高等数学、线性代数、概率论与数理统计提供了一套全方位、有特色、有成效的精品辅导书.这是几位教授几十年来教学经验与教学研究的总结与积累的成果。
    \end{xiaoti}
  本书题目的顺序安排基本上与教学同步,但是个别题目需要利用后续知识才能证明,之所以将这些题目前移,是为了便于将这些题目与前面相关题目进行对比,而这些题目都加了“*”,提请读者注意。

  本书适用于学习高等数学的理工类、学习微积分的经济类、管理类等本科学生学习使用,也适用于备考硕士研究生的学生选作复习参考书。

  本书得到北京航空航天大学教务处热情支持,并列入北京航空航天大学教材规划,为本书的完成起到有力的促进作用。
  \begin{flushright}
    作者于北京航空航天大学\\
    2006 年 12 月
  \end{flushright}
  \end{enumerate}
