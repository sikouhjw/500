\part{证明题}
	\chapter{极限与连续性}
	\section{极限}\label{sec:1.1.1}
	\begin{ti}
		设 $\lim_{n \to \infty} x_n = a > 0$,利用极限定义证明 $\lim_{n \to \infty} \frac{x_{n+1}}{x_n} = 1$。
	\end{ti}

	\begin{ti}
		设数列 $\{x_n\}$ 满足 $|x_{n+1}| \leq q|x_n|$ ($n = 1$, $2$, $\cdots$),其中 $0<q<1$。利用极限定义证明 $\lim_{n \to \infty} x_n = 0$。
		\begin{bianshi}
			设数列 $\{x_n\}$ 满足 $\lim_{n \to \infty} \bigl| \frac{x_{n+1}}{x_n} \bigr| = \lambda < 1$,证明 $\lim_{n \to \infty} x_n = 0$。
		\end{bianshi}
		\begin{bianshi}
			设 $x_n > 0$ ($n=1$, $2$, $\cdots$),且 $\lim_{n \to \infty} \frac{x_{n+1}}{x_n} = 0$,证明数列 $\{x_n\}$ 从某项起为单调减少。
		\end{bianshi}
	\end{ti}

	\begin{ti}
		利用极限定义证明:$\lim_{n \to \infty} x_n = A$ 的充分必要条件为 $\lim_{k \to \infty} x_{2k} = \lim_{k \to \infty} x_{2k+1} = A$。
	\end{ti}

	\begin{ti}
		利用极限定义证明:单调数列 $\{x_n\}$ 收敛于 $a$ 的充分必要条件是存在子数列 $\{x_{n_k}\}$ 收敛于 $a$。
	\end{ti}

	\begin{ti}
		设 $\lim_{n \to \infty} a_n = l$ ($l$ 为有限数或 $l = \pm \infty$)。证明
		\begin{xiaoti}
			\item $\lim_{n \to \infty} \frac{a_1 + a_2 + \cdots + a_n}{n} = l$;
			\item $\lim_{n \to \infty} \frac{a_1 + 2a_2 + \cdots + na_n}{1+2+\cdots+n} = l$;
			\item[(3)] \xing{} 当 $a_n > 0$ ($n=1$, $2$, $\cdots$) 时,$\lim_{n \to \infty} \sqrt[n]{a_1 \cdot a_2 \cdot \cdots \cdot a_n} = l$。
		\end{xiaoti}
	\end{ti}

	\begin{ti}
		试证下列数列 $\{x_n\}$ 存在极限,并求 $\lim_{n \to \infty} x_n$:
		\begin{xiaoti}
			\item $x_n = \sum_{k=1}^n \frac{k}{(k+1)!}$;
			\item $x_n = \sum_{k=1}^n \frac{1}{1+2+\cdots+k}$。
		\end{xiaoti}
	\end{ti}

	\begin{ti}
		设 $a_n = \cos \frac{\varphi}{2} \cdot \cos \frac{\varphi}{2^2} \cdot \cdots \cdot \cos \frac{\varphi}{2^n}$,试证数列 $\{a_n\}$ 存在极限,并求 $\lim_{n \to \infty} a_n$。
	\end{ti}

	\begin{ti}
		试证下列数列 $\{x_n\}$ 存在极限:
		\begin{xiaoti}
			\item $x_n = \bigl( 1 + \frac{1}{2} \bigr) \cdot \bigl( 1 + \frac{1}{2^2} \bigr) \cdot \cdots \cdot \bigl( 1 + \frac{1}{2^{2^n}} \bigr)$;
			\item $x_n = \bigl( 1 + \frac{1}{2} \bigr) \cdot \bigl( 1 + \frac{1}{2^2} \bigr) \cdot \cdots \cdot \bigl( 1 + \frac{1}{2^n} \bigr)$;
			\item $x_n = (1+a) \cdot \bigl( 1+a^2 \bigr) \cdot \cdots \cdot \bigl( 1 + a^{2^n} \bigr)$,其中 $|a| < 1$;
			\item $x_n = (1+a) \cdot \bigl( 1+a^2 \bigr) \cdot \cdots \cdot \bigl( 1 + a^n \bigr)$,其中 $|a| < 1$。
		\end{xiaoti}
	\end{ti}

	\begin{ti}
		\begin{xiaoti}
			\item 设 $a<b$, $x_0 = a$, $x_1 = b$ 及 $x_n = \frac{x_{n-1} + x_{n-2}}{2}$, $n=2$, $3$, $\cdots$。试证 $\lim_{n \to \infty} x_n = \frac{a+2b}{3}$;
			\item 设 $a_1 = 1$, $a_2 = 2$ 且 $a_{n+2} = \frac{2a_n a_{n+1}}{a_n + a_{n+1}}$ ($n=1$, $2$, $\cdots$)。证明 $\lim_{n \to \infty} a_n$ 存在,并求 $\lim_{n \to \infty} a_n$。
		\end{xiaoti}
	\end{ti}

	\begin{ti}
		设 $x_1 = 1$, $x_2 = 2$, $x_{n+2} = \sqrt{x_n x_{n+1}}$ ($n=1$, $2$, $\cdots$)。证明 $\lim_{n \to \infty} x_n$ 存在,并求 $\lim_{n \to \infty} x_n$。
	\end{ti}

	\begin{ti}
		证明下列数列 $\{x_n\}$ 存在极限,并求 $\lim_{n \to \infty} x_n$:
		\begin{xiaoti}
			\item $x_1 = 1$, $x_{n+1} = \sqrt{2x_n}$ ($n=1$, $2$, $\cdots$);
			\item $x_1 = \sqrt{2}$, $x_{n+1} = \sqrt{2+x_n}$ ($n=1$, $2$, $\cdots$)。
		\end{xiaoti}
	\end{ti}

	\begin{ti}
		设数列 $\{x_n\}$ 满足 $|x_{n+1} - x_n| \leq q^n$ ($n=1$, $2$, $\cdots$),其中 $0<q<1$。证明:$\lim_{n \to \infty} x_n$ 存在。
	\end{ti}

	\begin{ti}\label{sec:1.1.1-13}
		证明下列数列 $\{x_n\}$ 存在极限,并求 $\lim_{n \to \infty} x_n$:
		\begin{xiaoti}
			\item $x_1 = 10$, $x_{n+1} = \sqrt{6+x_n}$ ($n=1$, $2$, $\cdots$);
			\item $x_1 = 0$, $x_{n+1} = \sqrt{6+x_n}$ ($n=1$, $2$, $\cdots$);
			\item $x_1 > -6$, $x_{n+1} = \sqrt{6+x_n}$ ($n=1$, $2$, $\cdots$)。
		\end{xiaoti}
	\end{ti}

	\begin{ti}
		\begin{xiaoti}
			\item 设 $x_0 > 0$, $x_{n+1} = \frac{1}{1+x_n}$ ($n=0$, $1$, $2$, $\cdots$)。证明:$\lim_{n \to \infty} x_n$ 存在,并求 $\lim_{n \to \infty} x_n$;
			\item 设 $x_1 = 2$, $x_2 = 2 + \frac{1}{x_1}$, $\cdots$, $x_{n+1} = 2 + \frac{1}{x_n}$, $\cdots$。证明数列 $\{x_n\}$ 存在极限,并求 $\lim_{n \to \infty} x_n$。
		\end{xiaoti}
	\end{ti}

	\begin{ti}
		设 $0<x_1<3$, $x_{n+1} = \sqrt{x_n(3-x_n)}$ ($n=1$, $2$, $\cdots$)。证明数列 $\{x_n\}$ 的极限存在,并求此极限。
	\end{ti}

	\begin{ti}
		设 $A>0$, $x_1>0$, $x_{n+1} = \frac{1}{2} \bigl( x_n + \frac{A}{x_n} \bigr)$ ($n=1$, $2$, $\cdots$)。试证 $\lim_{n \to \infty} x_n$ 存在,并求 $\lim_{n \to \infty} x_n$。
	\end{ti}

	\begin{ti}
		证明下列数列 $\{x_n\}$ 存在极限,并求 $\lim_{n \to \infty} x_n$:
		\begin{xiaoti}
			\item $x_1>0$ 且 $x_{n+1} = \frac{3(1+x_n)}{3+x_n}$ ($n=1$, $2$, $\cdots$);
			\item $x_1>0$ 且 $x_{n+1} = \frac{A(1+x_n)}{A+x_n}$ ($n=1$, $2$, $\cdots$),其中 $A>0$。
		\end{xiaoti}
	\end{ti}

	\begin{ti}
		设 $0<x_n<1$, $x_{n+1}^2 = -x_n^2 + 2x_n$ ($n=1$, $2$, $\cdots$)。证明 $\lim_{n \to \infty} x_n$ 存在,并求 $\lim_{n \to \infty} x_n$。
	\end{ti}

	\begin{ti}
		设数列 $\{x_n\}$ 满足 $0<x_1<\uppi$, $x_{n+1} = \sin x_n$ ($n=1$, $2$, $\cdots$)。证明 $\lim_{n \to \infty} x_n$ 存在,并求该极限。
		\begin{bianshi}
			求 $\lim_{n \to \infty} \underbrace{\sin[\sin(\cdots \sin}_{(n \text{ 重})} x)]$。
		\end{bianshi}
	\end{ti}

	\begin{ti}
		设 $x_0 = m$, $x_{n+1} = m + \varepsilon \sin x_n$ ($n=0$, $1$, $2$, $\cdots$)。证明有 $\xi = \lim_{n \to \infty} x_n$,且数 $\xi$ 为方程
		\[ x - \varepsilon \sin x = m\quad (0<\varepsilon<1) \]
		的唯一根。
	\end{ti}

	\begin{ti}
		证明下列数列 $\{y_n\}$ 存在极限,并求 $\lim_{n \to \infty} y_n$:
		\begin{xiaoti}
			\item $y_1 = \frac{x}{2}$ ($0 \leq x \leq 1$), $y_n = \frac{x}{2} + \frac{y_{n-1}^2}{2}$ ($n=2$, $3$, $\cdots$);
			\item $y_1 = \frac{x}{2}$ ($0 \leq x \leq 1$), $y_n = \frac{x}{2} - \frac{y_{n-1}^2}{2}$ ($n=2$, $3$, $\cdots$)。
		\end{xiaoti}
	\end{ti}

	\begin{ti}
		若 $x_1 = a > 0$, $y_1 = b > 0$ ($a < b$),且 $x_{n+1} = \sqrt{x_n y_n}$, $y_{n+1} = \frac{1}{2}(x_n+y_n)$。证明 $\lim_{n \to \infty} x_n = \lim_{n \to \infty} y_n$。
	\end{ti}

	\begin{ti}
		设 $a_k \geq 0$ ($k=1$, $2$, $\cdots$, $r$)。证明 $\lim_{n \to \infty} \sqrt[n]{a_1^n + a_2^n + \cdots + a_r^n}$ 存在,并求 $\lim_{n \to \infty} \sqrt[n]{a_1^n + a_2^n + \cdots + a_r^n}$。
	\end{ti}

	\begin{ti}
		证明下列数列 $\{x_n\}$ 存在极限,并求 $\lim_{n \to \infty} x_n$:
		\begin{xiaoti}
			\item $x_n = \sum_{i=1}^{n} \frac{i}{n^2+i}$;
			\item \xing{} $x_n = \sum_{i=1}^{n} \frac{n}{n^2+i^2}$;
			\item \xing{} $x_n = \sum_{i=1}^{n^2} \frac{n}{n^2+i^2}$。
		\end{xiaoti}
	\end{ti}

	\begin{ti}
		判定下列数列 $\{x_n\}$ 是否存在极限,如果存在极限,求 $\lim_{n \to \infty} x_n$:
		\begin{xiaoti}
			\item $x_n = \sum_{i=1}^{n} \frac{1}{\sqrt{n^2+i}}$;
			\item $x_n = \sum_{i=1}^{n^2} \frac{1}{\sqrt{n^2+i}}$;
			\item \xing{} $x_n = \sum_{i=1}^{n} \frac{1}{\sqrt{n^2+i^2}}$。
		\end{xiaoti}
	\end{ti}

	\begin{ti}\xing{}
		证明下列数列 $\{x_n\}$ 存在极限,并求 $\lim_{n \to \infty} x_n$:
		\begin{xiaoti}
			\item $x_n = \ln \sqrt[n]{\bigl( 1 + \frac{1}{n} \bigr) \bigl( 1 + \frac{2}{n} \bigr) \cdots \bigl( 1 + \frac{n}{n} \bigr)}$;
			\item $x_n = \ln \sqrt[n]{\bigl( 1 + \frac{1}{n} \bigr)^k \cdot \bigl( 1 + \frac{2}{n} \bigr)^k \cdots \bigl( 1 + \frac{n}{n} \bigr)^k}$, $k$ 为正整数。
		\end{xiaoti}
	\end{ti}

	\begin{ti}\xing{}
		证明下列数列 $\{x_n\}$ 存在极限,并求 $\lim_{n \to \infty} x_n$:
		\begin{xiaoti}
			\item $x_n = \sum_{i=1}^n \frac{1}{n} \sin \frac{i}{n} \uppi$;
			\item $x_n = \sum_{i=1}^n \frac{1}{n+\frac{i}{n}} \sin \frac{i}{n} \uppi$。
		\end{xiaoti}
	\end{ti}

	\begin{ti}
		设函数 $f(x)$ 定义于 $(a,+\infty)$ 上,且在每一个有限区间 $(a,b)$ 内是有界的。证明
		\[ \lim_{x \to +\infty} \frac{f(x)}{x} = \lim_{x \to +\infty} [f(x+1) - f(x)]. \]
	\end{ti}

	\begin{ti}
		设 $f(x)$ 对于 $(-\infty,+\infty)$ 内的任意两点 $x$, $y$ 恒有
		\[ |f(x) - f(y)| \leq q |x-y|, \]
		其中 $0<q<1$。任取 $x_0 \in (-\infty,+\infty)$,令 $x_n = f(x_{n-1})$ ($n=1$, $2$, $\cdots$)。证明 $\lim_{n \to \infty} x_n = x^*$ 存在,且 $f(x^*) = x^*$。
	\end{ti}

	\section{连续性}
	\begin{ti}
		设 $f(x)$ 在 $(-\infty,+\infty)$ 内有定义,$f(x)$ 在点 $x=0$ 处连续,且对一切实数 $x_1$, $x_2$ 有
		\[ f(x_1+x_2) = f(x_1) + f(x_2), \]
		试证 $f(x)$ 在 $(-\infty,+\infty)$ 内处处连续。
	\end{ti}

	\begin{ti}
		证明黎曼函数
		\[ f(x) = \begin{cases}
			\frac{1}{n}, & x=\frac{m}{n}, \text{ 其中 $m$ 和 $n$ 为互质数}, \\
			0, & x \text{ 为无理数},
		\end{cases} \]
		当 $x$ 取任一个有理数时是不连续的,而当 $x$ 取任一个无理数时是连续的。
	\end{ti}

	\begin{ti}
		若 $f(x)$, $g(x)$ 都是连续函数,试证:
		\begin{xiaoti}
			\item $|f(x)|$ 为连续函数;
			\item $F(x) = \min\{f(x),g(x)\}$ 为连续函数;
			\item $G(x) = \max\{f(x),g(x)\}$ 为连续函数。
		\end{xiaoti}
	\end{ti}

	\begin{ti}
		\begin{xiaoti}
			\item 证明单调有界函数的一切不连续点均为第一类间断点;
			\item 若单调有界函数 $f(x)$ 可以取到 $f(a)$ 与 $f(b)$ 之间的一切值,则 $f(x)$ 在 $[a,b]$ 上连续。
		\end{xiaoti}
	\end{ti}

	\begin{ti}
		设函数 $f(x)$ 在 $(0,1)$ 内有定义,且函数 $\ee^x f(x)$ 与 $\ee^{-f(x)}$ 在 $(0,1)$ 内都是单调增加函数。证明:$f(x)$ 在 $(0,1)$ 内为连续函数。
	\end{ti}

	\begin{ti}
		\begin{xiaoti}
			\item 设 $f(x)$ 在 $(0,+\infty)$ 内连续,且 $f\bigl(x^2\bigr) = f(x)$ ($x>0$)。试证 $f(x)$ 在 $(0,+\infty)$ 内为常数;
			\item 设 $f(x)$ 在其定义区间内连续,且在有理点处值为零,试证 $f(x)$ 在其定义区间内恒为零;
			\item 若 $f(x)$ 在 $(-\infty,+\infty)$ 内连续,且对 $x$, $y$ 的一切实数值满足
			\[ f(x+y) = f(x) + f(y). \]
			试证 $f(x)$ 在 $(-\infty,+\infty)$ 内为线性函数 $f(x) = ax$,其中 $a = f(1)$;
			\item 设 $f(x)$ 在 $(-\infty,+\infty)$ 内连续,且对 $x$, $y$ 的一切实数值满足
			\[ f(x+y) = f(x) \cdot f(y). \]
			试证 $f(x)$ 在 $(-\infty,+\infty)$ 内不恒等于零时,一定为指数函数 $f(x) = a^x$,其中 $a=f(1)$;
			\item 设 $f(x)$ 在 $(0,+\infty)$ 内连续,且对 $x$, $y$ 的一切正实数值满足
			\[ f(xy) = f(x) + f(y). \]
			试证 $f(x)$ 在 $(0,+\infty)$ 内不恒等于零时,一定为对数函数 $f(x) = \log_a x$,其中 $a$ 为正常数;
			\item 设 $f(x)$ 在 $(0,+\infty)$ 内连续,且对 $x$, $y$ 的一切正实数值满足
			\[ f(xy) = f(x) \cdot f(y). \]
			试证 $f(x)$ 在 $(0,+\infty)$ 内不恒等于零时,一定为幂函数 $f(x) = x^a$,其中 $a$ 为常数。
		\end{xiaoti}
		\begin{bianshi}
			设函数 $f(x)$ 在 $(0,+\infty)$ 内连续,对任意 $x$ 有 $f\bigl( x^2 \bigr) = f(x)$,且 $f(3) = 5$,求 $f(x)$。
		\end{bianshi}
	\end{ti}

	\begin{ti}
		\begin{xiaoti}
			\item 若 $f(x)$ 在 $[a,+\infty)$ 上连续,且 $\lim_{x \to +\infty} f(x) = A$, $A$ 为有限数。试证 $f(x)$ 在 $[a,+\infty)$ 上有界;
			\item 若 $f(x)$ 在 $(-\infty,+\infty)$ 上连续,且 $\lim_{x \to -\infty} f(x) = A$, $\lim_{x \to +\infty} f(x) = B$, $A$, $B$ 为有限数。试证 $f(x)$ 在 $(-\infty,+\infty)$ 上有界;
			\item 设 $f(x)$ 在 $(a,b)$ 内连续,且 $\lim_{x \to a^+} f(x) = A$, $\lim_{x \to b^-} f(x) = B$, $A$, $B$ 为有限数。试证 $f(x)$ 在 $(a,b)$ 内有界;
			\item 设 $f(x)$ 在 $(a,b)$ 内连续,且 $f\bigl( a^+ \bigr) = \lim_{x \to a^+} f(x) = A$, $f\bigl( b^- \bigr) = \lim_{x \to b^-} f(x) = B$, $A$, $B$ 为有限数。试证 $f(x)$ 可以取得介于 $f\bigl( a^+ \bigr)$ 与 $f\bigl( b^- \bigr)$ 之间所有数值(不包括 $f\bigl( a^+ \bigr)$ 与 $f\bigl( b^- \bigr)$)。
		\end{xiaoti}
	\end{ti}

	\begin{ti}
		\begin{xiaoti}
			\item 设 $f(x)$ 在 $(a,b)$ 内连续,$x_1$, $x_2$, $\cdots$, $x_n$ 为 $(a,b)$ 内任意 $n$ 个点。试证存在 $\xi \in (a,b)$,使
			\[ f(\xi) = \frac{1}{n} \sum_{i=1}^n f(x_i); \]
			\item 设 $f(x)$ 在 $(a,b)$ 内连续且值恒正,$x_1$, $x_2$, $\cdots$, $x_n$ 为 $(a,b)$ 内任意 $n$ 个点。试证存在 $\xi \in (a,b)$,使
			\[ f(\xi) = \sqrt[n]{f(x_1) \cdot f(x_2) \cdots f(x_n)}; \]
			\item 设 $f(x)$ 在 $(a,b)$ 内连续,$x_1$, $x_2$, $\cdots$, $x_n$ 为 $(a,b)$ 内任意 $n$ 个点,$t_1 + t_2 + \cdots \allowbreak + t_n = 1$, $t_i > 0$ ($i=1$, $2$, $\cdots$, $n$)。试证存在 $\xi \in (a,b)$,使
			\[ f(\xi) = \sum_{i=1}^n t_i f(x_i). \]
		\end{xiaoti}
	\end{ti}

	\begin{ti}
		设 $f(x)$ 在 $(-\infty,+\infty)$ 上连续,且 $f[f(x)] = x$。试证存在 $x_0$,使 $f(x_0) = x_0$。
	\end{ti}

	\begin{ti}
		\begin{xiaoti}
			\item 设 $f(x)$, $g(x)$ 在 $[a,b]$ 上连续,$f(a) > g(a)$, $f(b) < g(b)$。试证至少存在一点 $\xi \in (a,b)$,使
			\[ f(\xi) = g(\xi); \]
			\item 设 $f(x)$ 在 $[a,b]$ 上连续,$f(a) < a$, $f(b) > b$。试证至少存在一点 $\xi \in (a,b)$,使
			\[ f(\xi) = \xi; \]
			\item 设 $f(x)$ 在 $[0,1]$ 上连续,$f(0) = 0$, $f(1) = 1$。试证至少存在一点 $\xi \in (0,1)$,使
			\[ f(\xi) = 1 - \xi. \]
		\end{xiaoti}
	\end{ti}

	\begin{ti}
		\begin{xiaoti}
			\item 设 $f(x)$ 在 $[0,1]$ 上连续,且 $f(0) = f(1) = 0$。试证对任意一个实数 $l$ ($0<l<1$),必定存在 $x_0 \in [0,1]$,使
			\[ f(\xi) = f(x_0 + l); \]
			\item 设 $f(x)$ 在 $[0,n]$ 上连续($n$ 为自然数,$n \geq 2$),$f(0) = f(n)$。试证存在 $\xi$, $\xi + 1 \in [0,n]$,使
			\[ f(\xi) = f(\xi + 1); \]
			\item 设 $f(x)$ 在 $[0,1]$ 上连续,$f(0) = f(1)$。试证对任意自然数 $n \geq 2$,必定存在 $\xi = \xi(n) \in (0,1)$,使
			\[ f(\xi) = f\biggl( \xi + \frac{1}{n} \biggr). \]
		\end{xiaoti}
	\end{ti}

	\begin{ti}
		\begin{xiaoti}
			\item 设 $f(x)$ 在 $[0,2a]$ 上连续,且 $f(0) = f(2a)$。证明在区间 $[0,a]$ 上存在 $\xi$,使
			\[ f(\xi) = f(\xi + a); \]
			\item 设 $f(x)$ 在 $[a,b]$ 上连续,$f(a) = f(b)$。证明在区间 $[a,b]$ 上存在 $\xi$,使
			\[ f(\xi) = f \biggl( \xi + \frac{b-a}{2} \biggr). \]
		\end{xiaoti}
	\end{ti}

	\begin{ti}
		设 $f(x)$ 在 $[a,b]$ 上为单调增加,且取正值的连续函数 ($a>0$)。证明存在点 $\xi \in (a,b)$,使
		\[ a^2 f(b) + b^2 f(a) = 2 \xi^2 f(\xi). \]
	\end{ti}

	\begin{ti}
		证明方程 $x = a \sin x + b$ ($a>0$, $b>0$) 至少有一个不超过 $a+b$ 的正根。
	\end{ti}

	\begin{ti}
		证明方程 $\frac{1}{x-1} + \frac{2}{x-2} + \frac{3}{x-3} = 0$ 有两个实根,并判定它们的范围。
	\end{ti}

	\begin{ti}
		若 $f(x)$ 在 $[a,b]$ 上连续,且对于任何 $x \in [a,b]$,存在相应的 $y \in [a,b]$,使得 $|f(y)| \leq \frac{1}{2} |f(x)|$。则至少存在一点 $\xi \in [a,b]$,使得
		\[ f(\xi) = 0. \]
	\end{ti}

	\begin{ti}
		设 $f(x)$, $g(x)$ 为有界闭区间 $[a,b]$ 上的连续函数,且有数列 $\{ x_n \} \subset [a,b]$,使 $g(x_n) = f(x_{n+1})$, $n=1$, $2$, $\cdots$。证明:至少存在一点 $x_0 \in [a,b]$,使
		\[ f(x_0) = g(x_0). \]
	\end{ti}